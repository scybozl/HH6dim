\documentclass[a4paper]{article}
\usepackage{axodraw}
\usepackage{longtable}
\usepackage{listings}
\usepackage{amsmath}
\usepackage{makeidx}
\usepackage{hyperref}

\newcommand{\bra}[1]{\langle #1 \vert}
\newcommand{\brb}[1]{[ #1 \vert}
\newcommand{\kea}[1]{\vert #1 \rangle}
\newcommand{\keb}[1]{\vert #1 ]}
\newcommand{\Spaa}[1]{\langle #1 \rangle}
\newcommand{\Spab}[1]{\langle #1]}
\newcommand{\Spba}[1]{[ #1 \rangle}
\newcommand{\Spbb}[1]{[ #1 ]}

\allowdisplaybreaks[1]

\title{\texttt{GoSam 2.0.4}: ${g}{g}\rightarrow{H}{H}$}
\author{scyboz}
\date{2017-05-18 (13:57:12)}

\renewcommand{\indexname}{Index of all Loop Diagrams}

\makeindex
\begin{document}
\maketitle
\begin{abstract}
\noindent This process consists of 3 tree-level diagrams and 36 NLO diagrams. GoSam has identified 3 groups  of NLO diagrams by analyzing their one-loop integrals.
\end{abstract}
\newpage
\tableofcontents
\newpage

\section{Helicities}

\begin{longtable}[c]{r|cccc}
\bf{Index} &1&2&3&4\\
\hline
\endfirsthead
\bf{Index} &1&2&3&4\\
\hline
\endhead 
$0$& $-$& $-$& $0$& $0$\\
$1\rightarrow 0$& $+$& $-$& $0$& $0$\\
$2\rightarrow 0$& $-$& $+$& $0$& $0$\\
$3\rightarrow 0$& $+$& $+$& $0$& $0$\\
\end{longtable}
\section{Wave Functions}
In this section, we use $l_i=k_i$ for massless particles;
in spinors $\kea{i}$ (resp. $\keb{i}$) denote $\kea{l_i}$ (resp. $\keb{l_i}$).
For the massive particles we have:
\begin{align}
l_{3} &= k_{3} - \frac{mH^2}{%
      2 k_{3}\cdot k_{2}}k_{2}\\
l_{4} &= k_{4} - \frac{mH^2}{%
      2 k_{4}\cdot k_{2}}k_{2}
\end{align}

All helicity amplitudes are defined in terms of the following wave functions:
\begin{itemize}
\item $g(k_{1})$ 
% incoming vector particle
\begin{align}
\varepsilon^\mu_+(k_{1}) &=
   \frac{\Spab{2\vert\gamma^\mu\vert 1}}{%
   \sqrt{2}\Spaa{2\vert 1}}\\
\varepsilon^\mu_-(k_{1}) &=
   \frac{\Spba{2\vert\gamma^\mu\vert 1}}{%
   \sqrt{2}\Spbb{1\vert 2}}
\end{align}
\item $g(k_{2})$ 
% incoming vector particle
\begin{align}
\varepsilon^\mu_+(k_{2}) &=
   \frac{\Spab{1\vert\gamma^\mu\vert 2}}{%
   \sqrt{2}\Spaa{1\vert 2}}\\
\varepsilon^\mu_-(k_{2}) &=
   \frac{\Spba{1\vert\gamma^\mu\vert 2}}{%
   \sqrt{2}\Spbb{2\vert 1}}
\end{align}
\item $H(k_3)$ 
\begin{align}
% outgoing scalar particle
\epsilon(k_{3}) &= 1
\end{align}
\item $H(k_4)$ 
\begin{align}
% outgoing scalar particle
\epsilon(k_{4}) &= 1
\end{align}
\end{itemize}

%------------------------------------------------------------------------
\section{Colour Basis}
\begin{align}
\vert c_{1}\rangle &=g^{A_{1}}_{(1)}g^{A_{2}}_{(2)}\textrm{tr}\left\{T^{A_{2}}T^{A_{1}}\right\}
\end{align}


\section{Tree Diagrams}
\lstinputlisting[title={QGraf Setup},frame=tlrb]{../diagrams-0.log}

\begin{longtable}{cc}
\endfirsthead
\endhead
%---#[ tree diagram1:
\hbox{
\begin{minipage}{0.45\textwidth}
\begin{center}
% Diagram 1:
\begin{picture}(140,120)(-10,-10)
   \Gluon(102.4,85.4)(58.0,69.2){3}{9} % g-propagator
   \Text(103.5,88.2)[lb]{$g(k_{1})$}
   \Gluon(1.2,59.8)(58.0,69.2){3}{12} % g-propagator
   \Text(0.7,56.8)[rt]{$g(k_{2})$}
   \DashLine(68.3,31.7)(109.9,22.2){5} % H-propagator
   \Text(110.6,19.3)[lt]{$H(k_{3})$}
   \DashLine(68.3,31.7)(37.0,3.8){5} % H-propagator
   \Text(39.0,6.0)[rt]{$H(k_{4})$}
   \Vertex(58.0,69.2){3} % g-g-H vertex
   \Vertex(68.3,31.7){3} % H-H-H vertex
   \DashLine(68.3,31.7)(58.0,69.2){5} % H-propagator
   \Text(60.2,49.7)[rt]{$H$}
\end{picture}
\\
{\sl Diagram~1}
\end{center}
\end{minipage}}
%---#] tree diagram1:
&
%---#[ tree diagram2:
\hbox{
\begin{minipage}{0.45\textwidth}
\begin{center}
% Diagram 2:
\begin{picture}(140,120)(-10,-10)
   \Gluon(102.4,85.4)(58.0,69.2){3}{9} % g-propagator
   \Text(103.5,88.2)[lb]{$g(k_{1})$}
   \Gluon(109.9,22.2)(68.3,31.7){3}{9} % g-propagator
   \Text(109.2,25.1)[lt]{$g(k_{2})$}
   \DashLine(58.0,69.2)(1.2,59.8){5} % H-propagator
   \Text(1.6,62.7)[rt]{$H(k_{3})$}
   \DashLine(68.3,31.7)(37.0,3.8){5} % H-propagator
   \Text(39.0,6.0)[rt]{$H(k_{4})$}
   \Vertex(58.0,69.2){3} % g-g-H vertex
   \Vertex(68.3,31.7){3} % g-g-H vertex
   \Gluon(68.3,31.7)(58.0,69.2){3}{8} % g-propagator
   \Text(60.2,49.7)[rt]{$g$}
\end{picture}
\\
{\sl Diagram~2}
\end{center}
\end{minipage}}
%---#] tree diagram2:
\\
%---#[ tree diagram3:
\hbox{
\begin{minipage}{0.45\textwidth}
\begin{center}
% Diagram 3:
\begin{picture}(140,120)(-10,-10)
   \Gluon(102.4,85.4)(58.0,69.2){3}{9} % g-propagator
   \Text(103.5,88.2)[lb]{$g(k_{1})$}
   \Gluon(109.9,22.2)(68.3,31.7){3}{9} % g-propagator
   \Text(109.2,25.1)[lt]{$g(k_{2})$}
   \DashLine(68.3,31.7)(37.0,3.8){5} % H-propagator
   \Text(39.0,6.0)[rt]{$H(k_{3})$}
   \DashLine(58.0,69.2)(1.2,59.8){5} % H-propagator
   \Text(1.6,62.7)[rt]{$H(k_{4})$}
   \Vertex(58.0,69.2){3} % g-g-H vertex
   \Vertex(68.3,31.7){3} % g-g-H vertex
   \Gluon(68.3,31.7)(58.0,69.2){3}{8} % g-propagator
   \Text(60.2,49.7)[rt]{$g$}
\end{picture}
\\
{\sl Diagram~3}
\end{center}
\end{minipage}}
%---#] tree diagram3:
\end{longtable}



\section{One-Loop Diagrams}
\subsection*{General Information}
\lstinputlisting[title={QGraf Setup},frame=tlrb]{../diagrams-1.log}

Loop diagrams are grouped into sets of diagrams which share
loop-propagators. A loop integral can be written as
\begin{equation}
\int\!\frac{\mathrm{d}^nk}{i\pi^{\frac{n}{2}}}%
\frac{\mathcal{N}(q)}{\prod_{j=1}{N}\left[(k+r_j)^2-(m_j^2
   -i m_j\Gamma_j) + i\delta\right]}.
\end{equation}
For each group we list $r_j$, $m_j$ and $\Gamma_j$.
For $m_j$ and $\Gamma_j$ only non-vanishing symbols are listed.
Furthermore, we give the matrix $S$ which is defined as
\begin{equation}
S_{\alpha\beta} = (r_\alpha-r_\beta)^2
-(m_\alpha^2-im_\alpha\Gamma_\alpha)
-(m_\beta^2-im_\beta\Gamma_\beta).
\end{equation}
For each diagram we denote how the matrix $S^\prime$ for the specific diagram
is obtained from the original~$S$. The notation
\begin{equation}
S^\prime=S_{Q\to q^\prime}^{\{l_1,l_2,\ldots\}}
\end{equation}
means, that the rows and columns labeled by $l_1,l_2,\ldots$ should be
removed from $S$ (likewise $r_{l_1}, r_{l_2}, \ldots$ are removed from the
list of propagators) and $\mathcal{N}(q)$ has to be replaced by
$\mathcal{N}(q^\prime)$.
The maximum effective rank of a group is the rank that has to be passed
to \textsc{Samurai} if the whole group is reduced at once; this number
is calculated as
\begin{equation}
\max_{\text{diagrams}}\left\{(\text{rank of diagram})+
(\text{number of pinches})\right\}.
\end{equation}
Diagrams with massless closed quark lines are multiplied by a factor
$\mathtt{Nfrat}=\mathtt{Nf}/\mathtt{Nfgen}$. This multiplication is indicated
by the symbol $N_f$ following the rank. By default $\mathtt{Nfrat}$ evaluates
to one but can be changed by modifying $\mathtt{Nf}$ or $\mathtt{Nfgen}$ in the
model~file.


\subsection{Group~0 (4-Point)}
\subsubsection*{General Information}
The maximum effective rank in this group is~6.

\begin{subequations}
\begin{align}
r_{1} &= -k_{2}+k_{4}\\
r_{2} &= -k_{2}\\
r_{3} &= 0\\
r_{4} &= -k_{3}
\end{align}
\end{subequations}

\begin{equation}
S=\left(\begin{array}{cccc}
   0&
   S_{1,2}&
   S_{1,3}&
   0\\
   S_{2,1}&
   0&
   0&
   S_{2,4}\\
   S_{3,1}&
   0&
   0&
   S_{3,4}\\
   0&
   S_{4,2}&
   S_{4,3}&
   0\end{array}\right)
\end{equation}
\begin{subequations}
\begin{align}
   S_{1,2}&=m_H^2\\
   S_{1,3}&=-s_{23}-s_{12}+2m_H^2\\
   S_{2,4}&=s_{23}\\
   S_{3,4}&=m_H^2
\end{align}
\end{subequations}

\subsubsection{Diagrams (10)}\begin{longtable}{cc}
\endfirsthead
\endhead
%---#[ loop diagram3:
\index{Diagram0000000003=Diagram 3 (Group 0)}
\hbox{
\begin{minipage}{0.45\textwidth}
\begin{center}
% Diagram 3:
\begin{picture}(140,120)(-10,-10)
   \Gluon(17.6,14.6)(33.5,50.0){3}{8} % g-propagator
   \Text(14.8,13.4)[rt]{$g(k_{1})$}
   \Gluon(102.4,85.4)(86.5,50.0){3}{8} % g-propagator
   \Text(105.2,86.6)[lb]{$g(k_{2})$}
   \DashLine(33.5,50.0)(17.6,85.4){5} % H-propagator
   \Text(14.8,86.6)[rb]{$H(k_{3})$}
   \DashLine(86.5,50.0)(102.4,14.6){5} % H-propagator
   \Text(105.2,13.4)[lt]{$H(k_{4})$}
   \Vertex(33.5,50.0){3} % g-g-g-H vertex
   \Vertex(86.5,50.0){3} % g-g-g-H vertex
   \GlueArc(59,50)(26,0,180){3}{17} % g-propagator
   \Text(60.0,79.5)[rb]{$g$}
   \GlueArc(59,50)(26,-180, 0){3}{17} % g-propagator
   \Text(60.0,20.5)[rt]{$g$}
\end{picture}
\\
{\sl Diagram~3}\\
$S^\prime=S^{\{2,4\}}_{Q\to q-(k2-k4)}$, $\mathrm{rk}=2$
\end{center}
\end{minipage}}
&
%---#] loop diagram3:

%---#[ loop diagram5:
\index{Diagram0000000005=Diagram 5 (Group 0)}
\hbox{
\begin{minipage}{0.45\textwidth}
\begin{center}
% Diagram 5:
\begin{picture}(140,120)(-10,-10)
   \Gluon(17.6,14.6)(29.1,50.0){3}{7} % g-propagator
   \Text(14.7,13.7)[rt]{$g(k_{1})$}
   \Gluon(102.4,85.4)(90.9,50.0){3}{7} % g-propagator
   \Text(105.3,86.3)[lb]{$g(k_{2})$}
   \DashLine(29.1,50.0)(17.6,85.4){5} % H-propagator
   \Text(14.7,86.3)[rb]{$H(k_{3})$}
   \DashLine(90.9,50.0)(102.4,14.6){5} % H-propagator
   \Text(105.3,13.7)[lt]{$H(k_{4})$}
   \Vertex(29.1,50.0){3} % g-g-H vertex
   \Vertex(52.3,50.0){3} % g-g-g vertex
   \Vertex(90.9,50.0){3} % g-g-g-H vertex
   \Gluon(52.3,50.0)(29.1,50.0){3}{5} % g-propagator
   \Text(40.7,47.0)[rt]{$g$}
   \GlueArc(71,49)(19,0,180){3}{12} % g-propagator
   \Text(71.6,72.3)[rb]{$g$}
   \GlueArc(71,49)(19,-180, 0){3}{12} % g-propagator
   \Text(71.6,27.7)[rt]{$g$}
\end{picture}
\\
{\sl Diagram~5}\\
$S^\prime=S^{\{2,4\}}_{Q\to q-(k2-k4)}$, $\mathrm{rk}=2$
\end{center}
\end{minipage}}
\\
%---#] loop diagram5:

%---#[ loop diagram8:
\index{Diagram0000000008=Diagram 8 (Group 0)}
\hbox{
\begin{minipage}{0.45\textwidth}
\begin{center}
% Diagram 8:
\begin{picture}(140,120)(-10,-10)
   \Gluon(102.4,85.4)(90.9,50.0){3}{7} % g-propagator
   \Text(105.3,86.3)[lb]{$g(k_{1})$}
   \Gluon(17.6,14.6)(29.1,50.0){3}{7} % g-propagator
   \Text(14.7,13.7)[rt]{$g(k_{2})$}
   \DashLine(90.9,50.0)(102.4,14.6){5} % H-propagator
   \Text(105.3,13.7)[lt]{$H(k_{3})$}
   \DashLine(29.1,50.0)(17.6,85.4){5} % H-propagator
   \Text(14.7,86.3)[rb]{$H(k_{4})$}
   \Vertex(29.1,50.0){3} % g-g-H vertex
   \Vertex(52.3,50.0){3} % g-g-g vertex
   \Vertex(90.9,50.0){3} % g-g-g-H vertex
   \Gluon(52.3,50.0)(29.1,50.0){3}{5} % g-propagator
   \Text(40.7,47.0)[rt]{$g$}
   \GlueArc(71,49)(19,0,180){3}{12} % g-propagator
   \Text(71.6,72.3)[rb]{$g$}
   \GlueArc(71,49)(19,-180, 0){3}{12} % g-propagator
   \Text(71.6,27.7)[rt]{$g$}
\end{picture}
\\
{\sl Diagram~8}\\
$S^\prime=S^{\{2,4\}}_{Q\to -q-(-k2+k4)}$, $\mathrm{rk}=2$
\end{center}
\end{minipage}}
&
%---#] loop diagram8:

%---#[ loop diagram10:
\index{Diagram0000000010=Diagram 10 (Group 0)}
\hbox{
\begin{minipage}{0.45\textwidth}
\begin{center}
% Diagram 10:
\begin{picture}(140,120)(-10,-10)
   \Gluon(102.4,14.6)(81.2,35.9){3}{6} % g-propagator
   \Text(100.3,16.8)[lt]{$g(k_{1})$}
   \Gluon(17.6,14.6)(38.8,50.0){3}{8} % g-propagator
   \Text(15.0,13.1)[rt]{$g(k_{2})$}
   \DashLine(81.2,64.1)(102.4,85.4){5} % H-propagator
   \Text(100.3,83.2)[lb]{$H(k_{3})$}
   \DashLine(38.8,50.0)(17.6,85.4){5} % H-propagator
   \Text(15.0,86.9)[rb]{$H(k_{4})$}
   \Vertex(81.2,35.9){3} % g-g-g vertex
   \Vertex(81.2,64.1){3} % g-g-H vertex
   \Vertex(38.8,50.0){3} % g-g-g-H vertex
   \Gluon(81.2,64.1)(81.2,35.9){3}{6} % g-propagator
   \Text(84.2,50.0)[lt]{$g$}
   \Gluon(38.8,50.0)(81.2,35.9){3}{9} % g-propagator
   \Text(60.9,45.8)[lb]{$g$}
   \Gluon(38.8,50.0)(81.2,64.1){3}{9} % g-propagator
   \Text(59.1,59.9)[rb]{$g$}
\end{picture}
\\
{\sl Diagram~10}\\
$S^\prime=S^{\{2\}}_{Q\to -q-(-k3)}$, $\mathrm{rk}=4$
\end{center}
\end{minipage}}
\\
%---#] loop diagram10:

%---#[ loop diagram13:
\index{Diagram0000000013=Diagram 13 (Group 0)}
\hbox{
\begin{minipage}{0.45\textwidth}
\begin{center}
% Diagram 13:
\begin{picture}(140,120)(-10,-10)
   \Gluon(17.6,14.6)(38.8,50.0){3}{8} % g-propagator
   \Text(15.0,13.1)[rt]{$g(k_{1})$}
   \Gluon(102.4,14.6)(81.2,35.9){3}{6} % g-propagator
   \Text(100.3,16.8)[lt]{$g(k_{2})$}
   \DashLine(38.8,50.0)(17.6,85.4){5} % H-propagator
   \Text(15.0,86.9)[rb]{$H(k_{3})$}
   \DashLine(81.2,64.1)(102.4,85.4){5} % H-propagator
   \Text(100.3,83.2)[lb]{$H(k_{4})$}
   \Vertex(81.2,35.9){3} % g-g-g vertex
   \Vertex(81.2,64.1){3} % g-g-H vertex
   \Vertex(38.8,50.0){3} % g-g-g-H vertex
   \Gluon(81.2,64.1)(81.2,35.9){3}{6} % g-propagator
   \Text(84.2,50.0)[lt]{$g$}
   \Gluon(38.8,50.0)(81.2,35.9){3}{9} % g-propagator
   \Text(60.9,45.8)[lb]{$g$}
   \Gluon(38.8,50.0)(81.2,64.1){3}{9} % g-propagator
   \Text(59.1,59.9)[rb]{$g$}
\end{picture}
\\
{\sl Diagram~13}\\
$S^\prime=S^{\{4\}}_{Q\to -q-(-k2)}$, $\mathrm{rk}=4$
\end{center}
\end{minipage}}
&
%---#] loop diagram13:

%---#[ loop diagram28:
\index{Diagram0000000028=Diagram 28 (Group 0)}
\hbox{
\begin{minipage}{0.45\textwidth}
\begin{center}
% Diagram 28:
\begin{picture}(140,120)(-10,-10)
   \Gluon(17.6,14.6)(26.7,50.0){3}{7} % g-propagator
   \Text(14.7,13.9)[rt]{$g(k_{1})$}
   \Gluon(102.4,85.4)(93.3,50.0){3}{7} % g-propagator
   \Text(105.3,86.1)[lb]{$g(k_{2})$}
   \DashLine(26.7,50.0)(17.6,85.4){5} % H-propagator
   \Text(14.7,86.1)[rb]{$H(k_{3})$}
   \DashLine(93.3,50.0)(102.4,14.6){5} % H-propagator
   \Text(105.3,13.9)[lt]{$H(k_{4})$}
   \Vertex(26.7,50.0){3} % g-g-H vertex
   \Vertex(93.3,50.0){3} % g-g-H vertex
   \Vertex(44.8,50.0){3} % ghbar-gh-g vertex
   \Vertex(75.2,50.0){3} % ghbar-gh-g vertex
   \Gluon(44.8,50.0)(26.7,50.0){3}{4} % g-propagator
   \Text(35.8,47.0)[rt]{$g$}
   \Gluon(75.2,50.0)(93.3,50.0){3}{4} % g-propagator
   \Text(84.2,53.0)[rb]{$g$}
   \DashArrowArc(59,49)(15,0,180){2} % ghbar-propagator
   \Text(60.0,68.2)[rb]{$u_g$}
   \DashArrowArc(59,49)(15,180,360){2} % ghbar-propagator
   \Text(60.0,31.8)[rt]{$u_g$}
\end{picture}
\\
{\sl -Diagram~28}\\
$S^\prime=S^{\{2,4\}}_{Q\to q-(k2-k4)}$, $\mathrm{rk}=2$
\end{center}
\end{minipage}}
\\
%---#] loop diagram28:

%---#[ loop diagram29:
\index{Diagram0000000029=Diagram 29 (Group 0)}
\hbox{
\begin{minipage}{0.45\textwidth}
\begin{center}
% Diagram 29:
\begin{picture}(140,120)(-10,-10)
   \Gluon(17.6,14.6)(26.7,50.0){3}{7} % g-propagator
   \Text(14.7,13.9)[rt]{$g(k_{1})$}
   \Gluon(102.4,85.4)(93.3,50.0){3}{7} % g-propagator
   \Text(105.3,86.1)[lb]{$g(k_{2})$}
   \DashLine(26.7,50.0)(17.6,85.4){5} % H-propagator
   \Text(14.7,86.1)[rb]{$H(k_{3})$}
   \DashLine(93.3,50.0)(102.4,14.6){5} % H-propagator
   \Text(105.3,13.9)[lt]{$H(k_{4})$}
   \Vertex(26.7,50.0){3} % g-g-H vertex
   \Vertex(93.3,50.0){3} % g-g-H vertex
   \Vertex(44.8,50.0){3} % g-g-g vertex
   \Vertex(75.2,50.0){3} % g-g-g vertex
   \Gluon(44.8,50.0)(26.7,50.0){3}{4} % g-propagator
   \Text(35.8,47.0)[rt]{$g$}
   \Gluon(75.2,50.0)(93.3,50.0){3}{4} % g-propagator
   \Text(84.2,53.0)[rb]{$g$}
   \GlueArc(59,49)(15,0,180){3}{10} % g-propagator
   \Text(60.0,68.2)[rb]{$g$}
   \GlueArc(59,49)(15,-180, 0){3}{10} % g-propagator
   \Text(60.0,31.8)[rt]{$g$}
\end{picture}
\\
{\sl Diagram~29}\\
$S^\prime=S^{\{2,4\}}_{Q\to q-(k2-k4)}$, $\mathrm{rk}=2$
\end{center}
\end{minipage}}
&
%---#] loop diagram29:

%---#[ loop diagram33:
\index{Diagram0000000033=Diagram 33 (Group 0)}
\hbox{
\begin{minipage}{0.45\textwidth}
\begin{center}
% Diagram 33:
\begin{picture}(140,120)(-10,-10)
   \Gluon(17.6,14.6)(31.7,50.0){3}{8} % g-propagator
   \Text(14.8,13.5)[rt]{$g(k_{1})$}
   \Gluon(102.4,14.6)(88.3,35.9){3}{5} % g-propagator
   \Text(99.9,16.3)[lt]{$g(k_{2})$}
   \DashLine(31.7,50.0)(17.6,85.4){5} % H-propagator
   \Text(14.8,86.5)[rb]{$H(k_{3})$}
   \DashLine(88.3,64.1)(102.4,85.4){5} % H-propagator
   \Text(99.9,83.7)[lb]{$H(k_{4})$}
   \Vertex(31.7,50.0){3} % g-g-H vertex
   \Vertex(88.3,35.9){3} % g-g-g vertex
   \Vertex(88.3,64.1){3} % g-g-H vertex
   \Vertex(60.0,50.0){3} % g-g-g vertex
   \Gluon(60.0,50.0)(31.7,50.0){3}{6} % g-propagator
   \Text(45.9,47.0)[rt]{$g$}
   \Gluon(88.3,64.1)(88.3,35.9){3}{6} % g-propagator
   \Text(91.3,50.0)[lt]{$g$}
   \Gluon(60.0,50.0)(88.3,35.9){3}{6} % g-propagator
   \Text(75.5,45.6)[lb]{$g$}
   \Gluon(60.0,50.0)(88.3,64.1){3}{6} % g-propagator
   \Text(72.8,59.8)[rb]{$g$}
\end{picture}
\\
{\sl Diagram~33}\\
$S^\prime=S^{\{4\}}_{Q\to -q-(-k2)}$, $\mathrm{rk}=4$
\end{center}
\end{minipage}}
\\
%---#] loop diagram33:

%---#[ loop diagram36:
\index{Diagram0000000036=Diagram 36 (Group 0)}
\hbox{
\begin{minipage}{0.45\textwidth}
\begin{center}
% Diagram 36:
\begin{picture}(140,120)(-10,-10)
   \Gluon(102.4,14.6)(88.3,35.9){3}{5} % g-propagator
   \Text(99.9,16.3)[lt]{$g(k_{1})$}
   \Gluon(17.6,14.6)(31.7,50.0){3}{8} % g-propagator
   \Text(14.8,13.5)[rt]{$g(k_{2})$}
   \DashLine(88.3,64.1)(102.4,85.4){5} % H-propagator
   \Text(99.9,83.7)[lb]{$H(k_{3})$}
   \DashLine(31.7,50.0)(17.6,85.4){5} % H-propagator
   \Text(14.8,86.5)[rb]{$H(k_{4})$}
   \Vertex(31.7,50.0){3} % g-g-H vertex
   \Vertex(88.3,35.9){3} % g-g-g vertex
   \Vertex(88.3,64.1){3} % g-g-H vertex
   \Vertex(60.0,50.0){3} % g-g-g vertex
   \Gluon(60.0,50.0)(31.7,50.0){3}{6} % g-propagator
   \Text(45.9,47.0)[rt]{$g$}
   \Gluon(88.3,64.1)(88.3,35.9){3}{6} % g-propagator
   \Text(91.3,50.0)[lt]{$g$}
   \Gluon(60.0,50.0)(88.3,35.9){3}{6} % g-propagator
   \Text(75.5,45.6)[lb]{$g$}
   \Gluon(60.0,50.0)(88.3,64.1){3}{6} % g-propagator
   \Text(72.8,59.8)[rb]{$g$}
\end{picture}
\\
{\sl Diagram~36}\\
$S^\prime=S^{\{2\}}_{Q\to -q-(-k3)}$, $\mathrm{rk}=4$
\end{center}
\end{minipage}}
&
%---#] loop diagram36:

%---#[ loop diagram40:
\index{Diagram0000000040=Diagram 40 (Group 0)}
\hbox{
\begin{minipage}{0.45\textwidth}
\begin{center}
% Diagram 40:
\begin{picture}(140,120)(-10,-10)
   \Gluon(102.4,85.4)(77.7,64.7){3}{6} % g-propagator
   \Text(104.3,87.7)[lb]{$g(k_{1})$}
   \Gluon(17.6,14.6)(42.3,35.3){3}{6} % g-propagator
   \Text(15.7,12.3)[rt]{$g(k_{2})$}
   \DashLine(77.7,35.3)(102.4,14.6){5} % H-propagator
   \Text(104.3,12.3)[lt]{$H(k_{3})$}
   \DashLine(42.3,64.7)(17.6,85.4){5} % H-propagator
   \Text(15.7,87.7)[rb]{$H(k_{4})$}
   \Vertex(77.7,64.7){3} % g-g-g vertex
   \Vertex(77.7,35.3){3} % g-g-H vertex
   \Vertex(42.3,64.7){3} % g-g-H vertex
   \Vertex(42.3,35.3){3} % g-g-g vertex
   \Gluon(77.7,35.3)(77.7,64.7){3}{6} % g-propagator
   \Text(74.7,50.0)[rt]{$g$}
   \Gluon(42.3,64.7)(77.7,64.7){3}{7} % g-propagator
   \Text(60.0,67.7)[rb]{$g$}
   \Gluon(42.3,35.3)(77.7,35.3){3}{7} % g-propagator
   \Text(60.0,38.3)[rb]{$g$}
   \Gluon(42.3,35.3)(42.3,64.7){3}{6} % g-propagator
   \Text(39.3,50.0)[rt]{$g$}
\end{picture}
\\
{\sl Diagram~40}\\
$S^\prime=S_{Q\to -q-(-k3)}$, $\mathrm{rk}=6$
\end{center}
\end{minipage}}

\end{longtable}

%---#] loop diagram40:

\subsection{Group~1 (4-Point)}
\subsubsection*{General Information}
The maximum effective rank in this group is~6.

\begin{subequations}
\begin{align}
r_{1} &= -k_{3}-k_{4}\\
r_{2} &= -k_{3}\\
r_{3} &= 0\\
r_{4} &= -k_{2}
\end{align}
\end{subequations}

\begin{equation}
S=\left(\begin{array}{cccc}
   0&
   S_{1,2}&
   S_{1,3}&
   0\\
   S_{2,1}&
   0&
   S_{2,3}&
   S_{2,4}\\
   S_{3,1}&
   S_{3,2}&
   0&
   0\\
   0&
   S_{4,2}&
   0&
   0\end{array}\right)
\end{equation}
\begin{subequations}
\begin{align}
   S_{1,2}&=m_H^2\\
   S_{1,3}&=s_{12}\\
   S_{2,3}&=m_H^2\\
   S_{2,4}&=s_{23}
\end{align}
\end{subequations}

\subsubsection{Diagrams (25)}\begin{longtable}{cc}
\endfirsthead
\endhead
%---#[ loop diagram1:
\index{Diagram0000000001=Diagram 1 (Group 1)}
\hbox{
\begin{minipage}{0.45\textwidth}
\begin{center}
% Diagram 1:
\begin{picture}(140,120)(-10,-10)
   \Gluon(102.4,85.4)(67.9,43.5){3}{11} % g-propagator
   \Text(104.7,87.3)[lb]{$g(k_{1})$}
   \Gluon(102.4,14.6)(67.9,43.5){3}{9} % g-propagator
   \Text(100.5,16.9)[lt]{$g(k_{2})$}
   \DashLine(36.4,69.6)(17.6,85.4){5} % H-propagator
   \Text(15.7,87.7)[rb]{$H(k_{3})$}
   \DashLine(67.9,43.5)(17.6,14.6){5} % H-propagator
   \Text(19.1,17.2)[rt]{$H(k_{4})$}
   \Vertex(36.4,69.6){3} % g-g-H vertex
   \Vertex(67.9,43.5){3} % g-g-g-g-H vertex
   \GlueArc(52,56)(20,-39,140){3}{13} % g-propagator
   \Text(67.2,74.6)[lb]{$g$}
   \GlueArc(52,56)(20,-219, -39){3}{13} % g-propagator
   \Text(37.1,38.5)[rt]{$g$}
\end{picture}
\\
{\sl Diagram~1}\\
$S^\prime=S^{\{1,4\}}_{Q\to q-(k3)}$, $\mathrm{rk}=2$
\end{center}
\end{minipage}}
&
%---#] loop diagram1:

%---#[ loop diagram2:
\index{Diagram0000000002=Diagram 2 (Group 1)}
\hbox{
\begin{minipage}{0.45\textwidth}
\begin{center}
% Diagram 2:
\begin{picture}(140,120)(-10,-10)
   \Gluon(102.4,85.4)(67.9,43.5){3}{11} % g-propagator
   \Text(104.7,87.3)[lb]{$g(k_{1})$}
   \Gluon(102.4,14.6)(67.9,43.5){3}{9} % g-propagator
   \Text(100.5,16.9)[lt]{$g(k_{2})$}
   \DashLine(67.9,43.5)(17.6,14.6){5} % H-propagator
   \Text(19.1,17.2)[rt]{$H(k_{3})$}
   \DashLine(36.4,69.6)(17.6,85.4){5} % H-propagator
   \Text(15.7,87.7)[rb]{$H(k_{4})$}
   \Vertex(36.4,69.6){3} % g-g-H vertex
   \Vertex(67.9,43.5){3} % g-g-g-g-H vertex
   \GlueArc(52,56)(20,-39,140){3}{13} % g-propagator
   \Text(67.2,74.6)[lb]{$g$}
   \GlueArc(52,56)(20,-219, -39){3}{13} % g-propagator
   \Text(37.1,38.5)[rt]{$g$}
\end{picture}
\\
{\sl Diagram~2}\\
$S^\prime=S^{\{3,4\}}_{Q\to q-(k3+k4)}$, $\mathrm{rk}=2$
\end{center}
\end{minipage}}
\\
%---#] loop diagram2:

%---#[ loop diagram4:
\index{Diagram0000000004=Diagram 4 (Group 1)}
\hbox{
\begin{minipage}{0.45\textwidth}
\begin{center}
% Diagram 4:
\begin{picture}(140,120)(-10,-10)
   \Gluon(17.6,14.6)(33.5,50.0){3}{8} % g-propagator
   \Text(14.8,13.4)[rt]{$g(k_{1})$}
   \Gluon(102.4,85.4)(86.5,50.0){3}{8} % g-propagator
   \Text(105.2,86.6)[lb]{$g(k_{2})$}
   \DashLine(86.5,50.0)(102.4,14.6){5} % H-propagator
   \Text(105.2,13.4)[lt]{$H(k_{3})$}
   \DashLine(33.5,50.0)(17.6,85.4){5} % H-propagator
   \Text(14.8,86.6)[rb]{$H(k_{4})$}
   \Vertex(33.5,50.0){3} % g-g-g-H vertex
   \Vertex(86.5,50.0){3} % g-g-g-H vertex
   \GlueArc(59,50)(26,0,180){3}{17} % g-propagator
   \Text(60.0,79.5)[rb]{$g$}
   \GlueArc(59,50)(26,-180, 0){3}{17} % g-propagator
   \Text(60.0,20.5)[rt]{$g$}
\end{picture}
\\
{\sl Diagram~4}\\
$S^\prime=S^{\{1,3\}}_{Q\to -q-(-k3)}$, $\mathrm{rk}=2$
\end{center}
\end{minipage}}
&
%---#] loop diagram4:

%---#[ loop diagram6:
\index{Diagram0000000006=Diagram 6 (Group 1)}
\hbox{
\begin{minipage}{0.45\textwidth}
\begin{center}
% Diagram 6:
\begin{picture}(140,120)(-10,-10)
   \Gluon(17.6,14.6)(29.1,50.0){3}{7} % g-propagator
   \Text(14.7,13.7)[rt]{$g(k_{1})$}
   \Gluon(102.4,85.4)(90.9,50.0){3}{7} % g-propagator
   \Text(105.3,86.3)[lb]{$g(k_{2})$}
   \DashLine(90.9,50.0)(102.4,14.6){5} % H-propagator
   \Text(105.3,13.7)[lt]{$H(k_{3})$}
   \DashLine(29.1,50.0)(17.6,85.4){5} % H-propagator
   \Text(14.7,86.3)[rb]{$H(k_{4})$}
   \Vertex(29.1,50.0){3} % g-g-H vertex
   \Vertex(52.3,50.0){3} % g-g-g vertex
   \Vertex(90.9,50.0){3} % g-g-g-H vertex
   \Gluon(52.3,50.0)(29.1,50.0){3}{5} % g-propagator
   \Text(40.7,47.0)[rt]{$g$}
   \GlueArc(71,49)(19,0,180){3}{12} % g-propagator
   \Text(71.6,72.3)[rb]{$g$}
   \GlueArc(71,49)(19,-180, 0){3}{12} % g-propagator
   \Text(71.6,27.7)[rt]{$g$}
\end{picture}
\\
{\sl Diagram~6}\\
$S^\prime=S^{\{1,3\}}_{Q\to -q-(-k3)}$, $\mathrm{rk}=2$
\end{center}
\end{minipage}}
\\
%---#] loop diagram6:

%---#[ loop diagram7:
\index{Diagram0000000007=Diagram 7 (Group 1)}
\hbox{
\begin{minipage}{0.45\textwidth}
\begin{center}
% Diagram 7:
\begin{picture}(140,120)(-10,-10)
   \Gluon(102.4,85.4)(90.9,50.0){3}{7} % g-propagator
   \Text(105.3,86.3)[lb]{$g(k_{1})$}
   \Gluon(17.6,14.6)(29.1,50.0){3}{7} % g-propagator
   \Text(14.7,13.7)[rt]{$g(k_{2})$}
   \DashLine(29.1,50.0)(17.6,85.4){5} % H-propagator
   \Text(14.7,86.3)[rb]{$H(k_{3})$}
   \DashLine(90.9,50.0)(102.4,14.6){5} % H-propagator
   \Text(105.3,13.7)[lt]{$H(k_{4})$}
   \Vertex(29.1,50.0){3} % g-g-H vertex
   \Vertex(52.3,50.0){3} % g-g-g vertex
   \Vertex(90.9,50.0){3} % g-g-g-H vertex
   \Gluon(52.3,50.0)(29.1,50.0){3}{5} % g-propagator
   \Text(40.7,47.0)[rt]{$g$}
   \GlueArc(71,49)(19,0,180){3}{12} % g-propagator
   \Text(71.6,72.3)[rb]{$g$}
   \GlueArc(71,49)(19,-180, 0){3}{12} % g-propagator
   \Text(71.6,27.7)[rt]{$g$}
\end{picture}
\\
{\sl Diagram~7}\\
$S^\prime=S^{\{1,3\}}_{Q\to q-(k3)}$, $\mathrm{rk}=2$
\end{center}
\end{minipage}}
&
%---#] loop diagram7:

%---#[ loop diagram9:
\index{Diagram0000000009=Diagram 9 (Group 1)}
\hbox{
\begin{minipage}{0.45\textwidth}
\begin{center}
% Diagram 9:
\begin{picture}(140,120)(-10,-10)
   \Gluon(102.4,85.4)(90.9,50.0){3}{7} % g-propagator
   \Text(105.3,86.3)[lb]{$g(k_{1})$}
   \Gluon(102.4,14.6)(90.9,50.0){3}{7} % g-propagator
   \Text(99.6,15.6)[lt]{$g(k_{2})$}
   \DashLine(29.1,50.0)(17.6,14.6){5} % H-propagator
   \Text(20.4,15.6)[rt]{$H(k_{3})$}
   \DashLine(29.1,50.0)(17.6,85.4){5} % H-propagator
   \Text(14.7,86.3)[rb]{$H(k_{4})$}
   \Vertex(29.1,50.0){3} % H-H-H vertex
   \Vertex(52.3,50.0){3} % g-g-H vertex
   \Vertex(90.9,50.0){3} % g-g-g-g vertex
   \DashLine(52.3,50.0)(29.1,50.0){5} % H-propagator
   \Text(40.7,47.0)[rt]{$H$}
   \GlueArc(71,49)(19,0,180){3}{12} % g-propagator
   \Text(71.6,72.3)[rb]{$g$}
   \GlueArc(71,49)(19,-180, 0){3}{12} % g-propagator
   \Text(71.6,27.7)[rt]{$g$}
\end{picture}
\\
{\sl Diagram~9}\\
$S^\prime=S^{\{2,4\}}_{Q\to q-(k3+k4)}$, $\mathrm{rk}=2$
\end{center}
\end{minipage}}
\\
%---#] loop diagram9:

%---#[ loop diagram11:
\index{Diagram0000000011=Diagram 11 (Group 1)}
\hbox{
\begin{minipage}{0.45\textwidth}
\begin{center}
% Diagram 11:
\begin{picture}(140,120)(-10,-10)
   \Gluon(102.4,14.6)(81.2,35.9){3}{6} % g-propagator
   \Text(100.3,16.8)[lt]{$g(k_{1})$}
   \Gluon(17.6,14.6)(38.8,50.0){3}{8} % g-propagator
   \Text(15.0,13.1)[rt]{$g(k_{2})$}
   \DashLine(38.8,50.0)(17.6,85.4){5} % H-propagator
   \Text(15.0,86.9)[rb]{$H(k_{3})$}
   \DashLine(81.2,64.1)(102.4,85.4){5} % H-propagator
   \Text(100.3,83.2)[lb]{$H(k_{4})$}
   \Vertex(81.2,35.9){3} % g-g-g vertex
   \Vertex(81.2,64.1){3} % g-g-H vertex
   \Vertex(38.8,50.0){3} % g-g-g-H vertex
   \Gluon(81.2,64.1)(81.2,35.9){3}{6} % g-propagator
   \Text(84.2,50.0)[lt]{$g$}
   \Gluon(38.8,50.0)(81.2,35.9){3}{9} % g-propagator
   \Text(60.9,45.8)[lb]{$g$}
   \Gluon(38.8,50.0)(81.2,64.1){3}{9} % g-propagator
   \Text(59.1,59.9)[rb]{$g$}
\end{picture}
\\
{\sl Diagram~11}\\
$S^\prime=S^{\{3\}}_{Q\to -q-(-k3-k4)}$, $\mathrm{rk}=4$
\end{center}
\end{minipage}}
&
%---#] loop diagram11:

%---#[ loop diagram12:
\index{Diagram0000000012=Diagram 12 (Group 1)}
\hbox{
\begin{minipage}{0.45\textwidth}
\begin{center}
% Diagram 12:
\begin{picture}(140,120)(-10,-10)
   \Gluon(17.6,14.6)(38.8,50.0){3}{8} % g-propagator
   \Text(15.0,13.1)[rt]{$g(k_{1})$}
   \Gluon(102.4,14.6)(81.2,35.9){3}{6} % g-propagator
   \Text(100.3,16.8)[lt]{$g(k_{2})$}
   \DashLine(81.2,64.1)(102.4,85.4){5} % H-propagator
   \Text(100.3,83.2)[lb]{$H(k_{3})$}
   \DashLine(38.8,50.0)(17.6,85.4){5} % H-propagator
   \Text(15.0,86.9)[rb]{$H(k_{4})$}
   \Vertex(81.2,35.9){3} % g-g-g vertex
   \Vertex(81.2,64.1){3} % g-g-H vertex
   \Vertex(38.8,50.0){3} % g-g-g-H vertex
   \Gluon(81.2,64.1)(81.2,35.9){3}{6} % g-propagator
   \Text(84.2,50.0)[lt]{$g$}
   \Gluon(38.8,50.0)(81.2,35.9){3}{9} % g-propagator
   \Text(60.9,45.8)[lb]{$g$}
   \Gluon(38.8,50.0)(81.2,64.1){3}{9} % g-propagator
   \Text(59.1,59.9)[rb]{$g$}
\end{picture}
\\
{\sl Diagram~12}\\
$S^\prime=S^{\{1\}}$, $\mathrm{rk}=4$
\end{center}
\end{minipage}}
\\
%---#] loop diagram12:

%---#[ loop diagram14:
\index{Diagram0000000014=Diagram 14 (Group 1)}
\hbox{
\begin{minipage}{0.45\textwidth}
\begin{center}
% Diagram 14:
\begin{picture}(140,120)(-10,-10)
   \Gluon(17.6,14.6)(38.8,50.0){3}{8} % g-propagator
   \Text(15.0,13.1)[rt]{$g(k_{1})$}
   \Gluon(17.6,85.4)(38.8,50.0){3}{8} % g-propagator
   \Text(20.1,83.8)[rb]{$g(k_{2})$}
   \DashLine(81.2,35.9)(102.4,14.6){5} % H-propagator
   \Text(104.5,12.5)[lt]{$H(k_{3})$}
   \DashLine(81.2,64.1)(102.4,85.4){5} % H-propagator
   \Text(100.3,83.2)[lb]{$H(k_{4})$}
   \Vertex(81.2,35.9){3} % g-g-H vertex
   \Vertex(81.2,64.1){3} % g-g-H vertex
   \Vertex(38.8,50.0){3} % g-g-g-g vertex
   \Gluon(81.2,64.1)(81.2,35.9){3}{6} % g-propagator
   \Text(84.2,50.0)[lt]{$g$}
   \Gluon(38.8,50.0)(81.2,35.9){3}{9} % g-propagator
   \Text(60.9,45.8)[lb]{$g$}
   \Gluon(38.8,50.0)(81.2,64.1){3}{9} % g-propagator
   \Text(59.1,59.9)[rb]{$g$}
\end{picture}
\\
{\sl Diagram~14}\\
$S^\prime=S^{\{4\}}_{Q\to q-(k3)}$, $\mathrm{rk}=4$
\end{center}
\end{minipage}}
&
%---#] loop diagram14:

%---#[ loop diagram17:
\index{Diagram0000000017=Diagram 17 (Group 1)}
\hbox{
\begin{minipage}{0.45\textwidth}
\begin{center}
% Diagram 17:
\begin{picture}(140,120)(-10,-10)
   \Gluon(102.4,85.4)(83.4,46.4){3}{9} % g-propagator
   \Text(105.1,86.7)[lb]{$g(k_{1})$}
   \Gluon(17.6,85.4)(28.0,68.0){3}{4} % g-propagator
   \Text(20.1,83.8)[rb]{$g(k_{2})$}
   \DashLine(83.4,46.4)(102.4,14.6){5} % H-propagator
   \Text(105.0,13.1)[lt]{$H(k_{3})$}
   \DashLine(45.3,39.2)(17.6,14.6){5} % H-propagator
   \Text(19.6,16.9)[rt]{$H(k_{4})$}
   \Vertex(83.4,46.4){3} % g-g-H vertex
   \Vertex(28.0,68.0){3} % g-g-g vertex
   \Vertex(45.3,39.2){3} % g-g-g-H vertex
   \Gluon(45.3,39.2)(83.4,46.4){3}{8} % g-propagator
   \Text(63.8,45.7)[rb]{$g$}
   \GlueArc(36,53)(16,-59,120){3}{11} % g-propagator
   \Text(53.6,63.8)[lb]{$g$}
   \GlueArc(36,53)(16,-239, -59){3}{11} % g-propagator
   \Text(19.6,43.4)[rt]{$g$}
\end{picture}
\\
{\sl Diagram~17}\\
$S^\prime=S^{\{1,2\}}$, $\mathrm{rk}=2$
\end{center}
\end{minipage}}
\\
%---#] loop diagram17:

%---#[ loop diagram18:
\index{Diagram0000000018=Diagram 18 (Group 1)}
\hbox{
\begin{minipage}{0.45\textwidth}
\begin{center}
% Diagram 18:
\begin{picture}(140,120)(-10,-10)
   \Gluon(102.4,85.4)(83.4,46.4){3}{9} % g-propagator
   \Text(105.1,86.7)[lb]{$g(k_{1})$}
   \Gluon(17.6,14.6)(45.3,39.2){3}{7} % g-propagator
   \Text(15.6,12.4)[rt]{$g(k_{2})$}
   \DashLine(83.4,46.4)(102.4,14.6){5} % H-propagator
   \Text(105.0,13.1)[lt]{$H(k_{3})$}
   \DashLine(28.0,68.0)(17.6,85.4){5} % H-propagator
   \Text(15.0,86.9)[rb]{$H(k_{4})$}
   \Vertex(83.4,46.4){3} % g-g-H vertex
   \Vertex(28.0,68.0){3} % g-g-H vertex
   \Vertex(45.3,39.2){3} % g-g-g-g vertex
   \Gluon(45.3,39.2)(83.4,46.4){3}{8} % g-propagator
   \Text(63.8,45.7)[rb]{$g$}
   \GlueArc(36,53)(16,-59,120){3}{11} % g-propagator
   \Text(53.6,63.8)[lb]{$g$}
   \GlueArc(36,53)(16,-239, -59){3}{11} % g-propagator
   \Text(19.6,43.4)[rt]{$g$}
\end{picture}
\\
{\sl Diagram~18}\\
$S^\prime=S^{\{3,4\}}_{Q\to q-(k3+k4)}$, $\mathrm{rk}=2$
\end{center}
\end{minipage}}
&
%---#] loop diagram18:

%---#[ loop diagram19:
\index{Diagram0000000019=Diagram 19 (Group 1)}
\hbox{
\begin{minipage}{0.45\textwidth}
\begin{center}
% Diagram 19:
\begin{picture}(140,120)(-10,-10)
   \Gluon(102.4,85.4)(83.4,46.4){3}{9} % g-propagator
   \Text(105.1,86.7)[lb]{$g(k_{1})$}
   \Gluon(17.6,85.4)(28.0,68.0){3}{4} % g-propagator
   \Text(20.1,83.8)[rb]{$g(k_{2})$}
   \DashLine(45.3,39.2)(17.6,14.6){5} % H-propagator
   \Text(19.6,16.9)[rt]{$H(k_{3})$}
   \DashLine(83.4,46.4)(102.4,14.6){5} % H-propagator
   \Text(105.0,13.1)[lt]{$H(k_{4})$}
   \Vertex(83.4,46.4){3} % g-g-H vertex
   \Vertex(28.0,68.0){3} % g-g-g vertex
   \Vertex(45.3,39.2){3} % g-g-g-H vertex
   \Gluon(45.3,39.2)(83.4,46.4){3}{8} % g-propagator
   \Text(63.8,45.7)[rb]{$g$}
   \GlueArc(36,53)(16,-59,120){3}{11} % g-propagator
   \Text(53.6,63.8)[lb]{$g$}
   \GlueArc(36,53)(16,-239, -59){3}{11} % g-propagator
   \Text(19.6,43.4)[rt]{$g$}
\end{picture}
\\
{\sl Diagram~19}\\
$S^\prime=S^{\{1,2\}}$, $\mathrm{rk}=2$
\end{center}
\end{minipage}}
\\
%---#] loop diagram19:

%---#[ loop diagram20:
\index{Diagram0000000020=Diagram 20 (Group 1)}
\hbox{
\begin{minipage}{0.45\textwidth}
\begin{center}
% Diagram 20:
\begin{picture}(140,120)(-10,-10)
   \Gluon(102.4,85.4)(83.4,46.4){3}{9} % g-propagator
   \Text(105.1,86.7)[lb]{$g(k_{1})$}
   \Gluon(17.6,14.6)(45.3,39.2){3}{7} % g-propagator
   \Text(15.6,12.4)[rt]{$g(k_{2})$}
   \DashLine(28.0,68.0)(17.6,85.4){5} % H-propagator
   \Text(15.0,86.9)[rb]{$H(k_{3})$}
   \DashLine(83.4,46.4)(102.4,14.6){5} % H-propagator
   \Text(105.0,13.1)[lt]{$H(k_{4})$}
   \Vertex(83.4,46.4){3} % g-g-H vertex
   \Vertex(28.0,68.0){3} % g-g-H vertex
   \Vertex(45.3,39.2){3} % g-g-g-g vertex
   \Gluon(45.3,39.2)(83.4,46.4){3}{8} % g-propagator
   \Text(63.8,45.7)[rb]{$g$}
   \GlueArc(36,53)(16,-59,120){3}{11} % g-propagator
   \Text(53.6,63.8)[lb]{$g$}
   \GlueArc(36,53)(16,-239, -59){3}{11} % g-propagator
   \Text(19.6,43.4)[rt]{$g$}
\end{picture}
\\
{\sl Diagram~20}\\
$S^\prime=S^{\{1,4\}}_{Q\to q-(k3)}$, $\mathrm{rk}=2$
\end{center}
\end{minipage}}
&
%---#] loop diagram20:

%---#[ loop diagram21:
\index{Diagram0000000021=Diagram 21 (Group 1)}
\hbox{
\begin{minipage}{0.45\textwidth}
\begin{center}
% Diagram 21:
\begin{picture}(140,120)(-10,-10)
   \Gluon(17.6,85.4)(28.0,68.0){3}{4} % g-propagator
   \Text(20.1,83.8)[rb]{$g(k_{1})$}
   \Gluon(102.4,85.4)(83.4,46.4){3}{9} % g-propagator
   \Text(105.1,86.7)[lb]{$g(k_{2})$}
   \DashLine(83.4,46.4)(102.4,14.6){5} % H-propagator
   \Text(105.0,13.1)[lt]{$H(k_{3})$}
   \DashLine(45.3,39.2)(17.6,14.6){5} % H-propagator
   \Text(19.6,16.9)[rt]{$H(k_{4})$}
   \Vertex(83.4,46.4){3} % g-g-H vertex
   \Vertex(28.0,68.0){3} % g-g-g vertex
   \Vertex(45.3,39.2){3} % g-g-g-H vertex
   \Gluon(45.3,39.2)(83.4,46.4){3}{8} % g-propagator
   \Text(63.8,45.7)[rb]{$g$}
   \GlueArc(36,53)(16,-59,120){3}{11} % g-propagator
   \Text(53.6,63.8)[lb]{$g$}
   \GlueArc(36,53)(16,-239, -59){3}{11} % g-propagator
   \Text(19.6,43.4)[rt]{$g$}
\end{picture}
\\
{\sl Diagram~21}\\
$S^\prime=S^{\{2,3\}}_{Q\to -q-(-k3-k4)}$, $\mathrm{rk}=2$
\end{center}
\end{minipage}}
\\
%---#] loop diagram21:

%---#[ loop diagram22:
\index{Diagram0000000022=Diagram 22 (Group 1)}
\hbox{
\begin{minipage}{0.45\textwidth}
\begin{center}
% Diagram 22:
\begin{picture}(140,120)(-10,-10)
   \Gluon(17.6,14.6)(45.3,39.2){3}{7} % g-propagator
   \Text(15.6,12.4)[rt]{$g(k_{1})$}
   \Gluon(102.4,85.4)(83.4,46.4){3}{9} % g-propagator
   \Text(105.1,86.7)[lb]{$g(k_{2})$}
   \DashLine(83.4,46.4)(102.4,14.6){5} % H-propagator
   \Text(105.0,13.1)[lt]{$H(k_{3})$}
   \DashLine(28.0,68.0)(17.6,85.4){5} % H-propagator
   \Text(15.0,86.9)[rb]{$H(k_{4})$}
   \Vertex(83.4,46.4){3} % g-g-H vertex
   \Vertex(28.0,68.0){3} % g-g-H vertex
   \Vertex(45.3,39.2){3} % g-g-g-g vertex
   \Gluon(45.3,39.2)(83.4,46.4){3}{8} % g-propagator
   \Text(63.8,45.7)[rb]{$g$}
   \GlueArc(36,53)(16,-59,120){3}{11} % g-propagator
   \Text(53.6,63.8)[lb]{$g$}
   \GlueArc(36,53)(16,-239, -59){3}{11} % g-propagator
   \Text(19.6,43.4)[rt]{$g$}
\end{picture}
\\
{\sl Diagram~22}\\
$S^\prime=S^{\{3,4\}}_{Q\to q-(k3+k4)}$, $\mathrm{rk}=2$
\end{center}
\end{minipage}}
&
%---#] loop diagram22:

%---#[ loop diagram23:
\index{Diagram0000000023=Diagram 23 (Group 1)}
\hbox{
\begin{minipage}{0.45\textwidth}
\begin{center}
% Diagram 23:
\begin{picture}(140,120)(-10,-10)
   \Gluon(17.6,85.4)(28.0,68.0){3}{4} % g-propagator
   \Text(20.1,83.8)[rb]{$g(k_{1})$}
   \Gluon(102.4,85.4)(83.4,46.4){3}{9} % g-propagator
   \Text(105.1,86.7)[lb]{$g(k_{2})$}
   \DashLine(45.3,39.2)(17.6,14.6){5} % H-propagator
   \Text(19.6,16.9)[rt]{$H(k_{3})$}
   \DashLine(83.4,46.4)(102.4,14.6){5} % H-propagator
   \Text(105.0,13.1)[lt]{$H(k_{4})$}
   \Vertex(83.4,46.4){3} % g-g-H vertex
   \Vertex(28.0,68.0){3} % g-g-g vertex
   \Vertex(45.3,39.2){3} % g-g-g-H vertex
   \Gluon(45.3,39.2)(83.4,46.4){3}{8} % g-propagator
   \Text(63.8,45.7)[rb]{$g$}
   \GlueArc(36,53)(16,-59,120){3}{11} % g-propagator
   \Text(53.6,63.8)[lb]{$g$}
   \GlueArc(36,53)(16,-239, -59){3}{11} % g-propagator
   \Text(19.6,43.4)[rt]{$g$}
\end{picture}
\\
{\sl Diagram~23}\\
$S^\prime=S^{\{2,3\}}_{Q\to -q-(-k3-k4)}$, $\mathrm{rk}=2$
\end{center}
\end{minipage}}
\\
%---#] loop diagram23:

%---#[ loop diagram24:
\index{Diagram0000000024=Diagram 24 (Group 1)}
\hbox{
\begin{minipage}{0.45\textwidth}
\begin{center}
% Diagram 24:
\begin{picture}(140,120)(-10,-10)
   \Gluon(17.6,14.6)(45.3,39.2){3}{7} % g-propagator
   \Text(15.6,12.4)[rt]{$g(k_{1})$}
   \Gluon(102.4,85.4)(83.4,46.4){3}{9} % g-propagator
   \Text(105.1,86.7)[lb]{$g(k_{2})$}
   \DashLine(28.0,68.0)(17.6,85.4){5} % H-propagator
   \Text(15.0,86.9)[rb]{$H(k_{3})$}
   \DashLine(83.4,46.4)(102.4,14.6){5} % H-propagator
   \Text(105.0,13.1)[lt]{$H(k_{4})$}
   \Vertex(83.4,46.4){3} % g-g-H vertex
   \Vertex(28.0,68.0){3} % g-g-H vertex
   \Vertex(45.3,39.2){3} % g-g-g-g vertex
   \Gluon(45.3,39.2)(83.4,46.4){3}{8} % g-propagator
   \Text(63.8,45.7)[rb]{$g$}
   \GlueArc(36,53)(16,-59,120){3}{11} % g-propagator
   \Text(53.6,63.8)[lb]{$g$}
   \GlueArc(36,53)(16,-239, -59){3}{11} % g-propagator
   \Text(19.6,43.4)[rt]{$g$}
\end{picture}
\\
{\sl Diagram~24}\\
$S^\prime=S^{\{1,4\}}_{Q\to q-(k3)}$, $\mathrm{rk}=2$
\end{center}
\end{minipage}}
&
%---#] loop diagram24:

%---#[ loop diagram25:
\index{Diagram0000000025=Diagram 25 (Group 1)}
\hbox{
\begin{minipage}{0.45\textwidth}
\begin{center}
% Diagram 25:
\begin{picture}(140,120)(-10,-10)
   \Gluon(17.6,85.4)(28.0,68.0){3}{4} % g-propagator
   \Text(20.1,83.8)[rb]{$g(k_{1})$}
   \Gluon(17.6,14.6)(45.3,39.2){3}{7} % g-propagator
   \Text(15.6,12.4)[rt]{$g(k_{2})$}
   \DashLine(83.4,46.4)(102.4,85.4){5} % H-propagator
   \Text(99.7,84.0)[lb]{$H(k_{3})$}
   \DashLine(83.4,46.4)(102.4,14.6){5} % H-propagator
   \Text(105.0,13.1)[lt]{$H(k_{4})$}
   \Vertex(83.4,46.4){3} % H-H-H vertex
   \Vertex(28.0,68.0){3} % g-g-g vertex
   \Vertex(45.3,39.2){3} % g-g-g-H vertex
   \DashLine(45.3,39.2)(83.4,46.4){5} % H-propagator
   \Text(63.8,45.7)[rb]{$H$}
   \GlueArc(36,53)(16,-59,120){3}{11} % g-propagator
   \Text(53.6,63.8)[lb]{$g$}
   \GlueArc(36,53)(16,-239, -59){3}{11} % g-propagator
   \Text(19.6,43.4)[rt]{$g$}
\end{picture}
\\
{\sl Diagram~25}\\
$S^\prime=S^{\{2,3\}}_{Q\to -q-(-k3-k4)}$, $\mathrm{rk}=2$
\end{center}
\end{minipage}}
\\
%---#] loop diagram25:

%---#[ loop diagram26:
\index{Diagram0000000026=Diagram 26 (Group 1)}
\hbox{
\begin{minipage}{0.45\textwidth}
\begin{center}
% Diagram 26:
\begin{picture}(140,120)(-10,-10)
   \Gluon(17.6,14.6)(45.3,39.2){3}{7} % g-propagator
   \Text(15.6,12.4)[rt]{$g(k_{1})$}
   \Gluon(17.6,85.4)(28.0,68.0){3}{4} % g-propagator
   \Text(20.1,83.8)[rb]{$g(k_{2})$}
   \DashLine(83.4,46.4)(102.4,85.4){5} % H-propagator
   \Text(99.7,84.0)[lb]{$H(k_{3})$}
   \DashLine(83.4,46.4)(102.4,14.6){5} % H-propagator
   \Text(105.0,13.1)[lt]{$H(k_{4})$}
   \Vertex(83.4,46.4){3} % H-H-H vertex
   \Vertex(28.0,68.0){3} % g-g-g vertex
   \Vertex(45.3,39.2){3} % g-g-g-H vertex
   \DashLine(45.3,39.2)(83.4,46.4){5} % H-propagator
   \Text(63.8,45.7)[rb]{$H$}
   \GlueArc(36,53)(16,-59,120){3}{11} % g-propagator
   \Text(53.6,63.8)[lb]{$g$}
   \GlueArc(36,53)(16,-239, -59){3}{11} % g-propagator
   \Text(19.6,43.4)[rt]{$g$}
\end{picture}
\\
{\sl Diagram~26}\\
$S^\prime=S^{\{1,2\}}$, $\mathrm{rk}=2$
\end{center}
\end{minipage}}
&
%---#] loop diagram26:

%---#[ loop diagram30:
\index{Diagram0000000030=Diagram 30 (Group 1)}
\hbox{
\begin{minipage}{0.45\textwidth}
\begin{center}
% Diagram 30:
\begin{picture}(140,120)(-10,-10)
   \Gluon(17.6,14.6)(26.7,50.0){3}{7} % g-propagator
   \Text(14.7,13.9)[rt]{$g(k_{1})$}
   \Gluon(102.4,85.4)(93.3,50.0){3}{7} % g-propagator
   \Text(105.3,86.1)[lb]{$g(k_{2})$}
   \DashLine(93.3,50.0)(102.4,14.6){5} % H-propagator
   \Text(105.3,13.9)[lt]{$H(k_{3})$}
   \DashLine(26.7,50.0)(17.6,85.4){5} % H-propagator
   \Text(14.7,86.1)[rb]{$H(k_{4})$}
   \Vertex(26.7,50.0){3} % g-g-H vertex
   \Vertex(93.3,50.0){3} % g-g-H vertex
   \Vertex(44.8,50.0){3} % ghbar-gh-g vertex
   \Vertex(75.2,50.0){3} % ghbar-gh-g vertex
   \Gluon(44.8,50.0)(26.7,50.0){3}{4} % g-propagator
   \Text(35.8,47.0)[rt]{$g$}
   \Gluon(75.2,50.0)(93.3,50.0){3}{4} % g-propagator
   \Text(84.2,53.0)[rb]{$g$}
   \DashArrowArc(59,49)(15,0,180){2} % ghbar-propagator
   \Text(60.0,68.2)[rb]{$u_g$}
   \DashArrowArc(59,49)(15,180,360){2} % ghbar-propagator
   \Text(60.0,31.8)[rt]{$u_g$}
\end{picture}
\\
{\sl -Diagram~30}\\
$S^\prime=S^{\{1,3\}}_{Q\to -q-(-k3)}$, $\mathrm{rk}=2$
\end{center}
\end{minipage}}
\\
%---#] loop diagram30:

%---#[ loop diagram31:
\index{Diagram0000000031=Diagram 31 (Group 1)}
\hbox{
\begin{minipage}{0.45\textwidth}
\begin{center}
% Diagram 31:
\begin{picture}(140,120)(-10,-10)
   \Gluon(17.6,14.6)(26.7,50.0){3}{7} % g-propagator
   \Text(14.7,13.9)[rt]{$g(k_{1})$}
   \Gluon(102.4,85.4)(93.3,50.0){3}{7} % g-propagator
   \Text(105.3,86.1)[lb]{$g(k_{2})$}
   \DashLine(93.3,50.0)(102.4,14.6){5} % H-propagator
   \Text(105.3,13.9)[lt]{$H(k_{3})$}
   \DashLine(26.7,50.0)(17.6,85.4){5} % H-propagator
   \Text(14.7,86.1)[rb]{$H(k_{4})$}
   \Vertex(26.7,50.0){3} % g-g-H vertex
   \Vertex(93.3,50.0){3} % g-g-H vertex
   \Vertex(44.8,50.0){3} % g-g-g vertex
   \Vertex(75.2,50.0){3} % g-g-g vertex
   \Gluon(44.8,50.0)(26.7,50.0){3}{4} % g-propagator
   \Text(35.8,47.0)[rt]{$g$}
   \Gluon(75.2,50.0)(93.3,50.0){3}{4} % g-propagator
   \Text(84.2,53.0)[rb]{$g$}
   \GlueArc(59,49)(15,0,180){3}{10} % g-propagator
   \Text(60.0,68.2)[rb]{$g$}
   \GlueArc(59,49)(15,-180, 0){3}{10} % g-propagator
   \Text(60.0,31.8)[rt]{$g$}
\end{picture}
\\
{\sl Diagram~31}\\
$S^\prime=S^{\{1,3\}}_{Q\to -q-(-k3)}$, $\mathrm{rk}=2$
\end{center}
\end{minipage}}
&
%---#] loop diagram31:

%---#[ loop diagram34:
\index{Diagram0000000034=Diagram 34 (Group 1)}
\hbox{
\begin{minipage}{0.45\textwidth}
\begin{center}
% Diagram 34:
\begin{picture}(140,120)(-10,-10)
   \Gluon(17.6,14.6)(31.7,50.0){3}{8} % g-propagator
   \Text(14.8,13.5)[rt]{$g(k_{1})$}
   \Gluon(102.4,14.6)(88.3,35.9){3}{5} % g-propagator
   \Text(99.9,16.3)[lt]{$g(k_{2})$}
   \DashLine(88.3,64.1)(102.4,85.4){5} % H-propagator
   \Text(99.9,83.7)[lb]{$H(k_{3})$}
   \DashLine(31.7,50.0)(17.6,85.4){5} % H-propagator
   \Text(14.8,86.5)[rb]{$H(k_{4})$}
   \Vertex(31.7,50.0){3} % g-g-H vertex
   \Vertex(88.3,35.9){3} % g-g-g vertex
   \Vertex(88.3,64.1){3} % g-g-H vertex
   \Vertex(60.0,50.0){3} % g-g-g vertex
   \Gluon(60.0,50.0)(31.7,50.0){3}{6} % g-propagator
   \Text(45.9,47.0)[rt]{$g$}
   \Gluon(88.3,64.1)(88.3,35.9){3}{6} % g-propagator
   \Text(91.3,50.0)[lt]{$g$}
   \Gluon(60.0,50.0)(88.3,35.9){3}{6} % g-propagator
   \Text(75.5,45.6)[lb]{$g$}
   \Gluon(60.0,50.0)(88.3,64.1){3}{6} % g-propagator
   \Text(72.8,59.8)[rb]{$g$}
\end{picture}
\\
{\sl Diagram~34}\\
$S^\prime=S^{\{1\}}$, $\mathrm{rk}=4$
\end{center}
\end{minipage}}
\\
%---#] loop diagram34:

%---#[ loop diagram35:
\index{Diagram0000000035=Diagram 35 (Group 1)}
\hbox{
\begin{minipage}{0.45\textwidth}
\begin{center}
% Diagram 35:
\begin{picture}(140,120)(-10,-10)
   \Gluon(102.4,14.6)(88.3,35.9){3}{5} % g-propagator
   \Text(99.9,16.3)[lt]{$g(k_{1})$}
   \Gluon(17.6,14.6)(31.7,50.0){3}{8} % g-propagator
   \Text(14.8,13.5)[rt]{$g(k_{2})$}
   \DashLine(31.7,50.0)(17.6,85.4){5} % H-propagator
   \Text(14.8,86.5)[rb]{$H(k_{3})$}
   \DashLine(88.3,64.1)(102.4,85.4){5} % H-propagator
   \Text(99.9,83.7)[lb]{$H(k_{4})$}
   \Vertex(31.7,50.0){3} % g-g-H vertex
   \Vertex(88.3,35.9){3} % g-g-g vertex
   \Vertex(88.3,64.1){3} % g-g-H vertex
   \Vertex(60.0,50.0){3} % g-g-g vertex
   \Gluon(60.0,50.0)(31.7,50.0){3}{6} % g-propagator
   \Text(45.9,47.0)[rt]{$g$}
   \Gluon(88.3,64.1)(88.3,35.9){3}{6} % g-propagator
   \Text(91.3,50.0)[lt]{$g$}
   \Gluon(60.0,50.0)(88.3,35.9){3}{6} % g-propagator
   \Text(75.5,45.6)[lb]{$g$}
   \Gluon(60.0,50.0)(88.3,64.1){3}{6} % g-propagator
   \Text(72.8,59.8)[rb]{$g$}
\end{picture}
\\
{\sl Diagram~35}\\
$S^\prime=S^{\{3\}}_{Q\to -q-(-k3-k4)}$, $\mathrm{rk}=4$
\end{center}
\end{minipage}}
&
%---#] loop diagram35:

%---#[ loop diagram37:
\index{Diagram0000000037=Diagram 37 (Group 1)}
\hbox{
\begin{minipage}{0.45\textwidth}
\begin{center}
% Diagram 37:
\begin{picture}(140,120)(-10,-10)
   \Gluon(102.4,14.6)(88.3,35.9){3}{5} % g-propagator
   \Text(99.9,16.3)[lt]{$g(k_{1})$}
   \Gluon(102.4,85.4)(88.3,64.1){3}{5} % g-propagator
   \Text(104.9,87.0)[lb]{$g(k_{2})$}
   \DashLine(31.7,50.0)(17.6,14.6){5} % H-propagator
   \Text(20.4,15.8)[rt]{$H(k_{3})$}
   \DashLine(31.7,50.0)(17.6,85.4){5} % H-propagator
   \Text(14.8,86.5)[rb]{$H(k_{4})$}
   \Vertex(31.7,50.0){3} % H-H-H vertex
   \Vertex(88.3,35.9){3} % g-g-g vertex
   \Vertex(88.3,64.1){3} % g-g-g vertex
   \Vertex(60.0,50.0){3} % g-g-H vertex
   \DashLine(60.0,50.0)(31.7,50.0){5} % H-propagator
   \Text(45.9,47.0)[rt]{$H$}
   \Gluon(88.3,64.1)(88.3,35.9){3}{6} % g-propagator
   \Text(91.3,50.0)[lt]{$g$}
   \Gluon(60.0,50.0)(88.3,35.9){3}{6} % g-propagator
   \Text(75.5,45.6)[lb]{$g$}
   \Gluon(60.0,50.0)(88.3,64.1){3}{6} % g-propagator
   \Text(72.8,59.8)[rb]{$g$}
\end{picture}
\\
{\sl Diagram~37}\\
$S^\prime=S^{\{2\}}_{Q\to q-(k2)}$, $\mathrm{rk}=4$
\end{center}
\end{minipage}}
\\
%---#] loop diagram37:

%---#[ loop diagram39:
\index{Diagram0000000039=Diagram 39 (Group 1)}
\hbox{
\begin{minipage}{0.45\textwidth}
\begin{center}
% Diagram 39:
\begin{picture}(140,120)(-10,-10)
   \Gluon(102.4,85.4)(77.7,64.7){3}{6} % g-propagator
   \Text(104.3,87.7)[lb]{$g(k_{1})$}
   \Gluon(102.4,14.6)(77.7,35.3){3}{6} % g-propagator
   \Text(100.5,16.9)[lt]{$g(k_{2})$}
   \DashLine(42.3,35.3)(17.6,14.6){5} % H-propagator
   \Text(19.5,16.9)[rt]{$H(k_{3})$}
   \DashLine(42.3,64.7)(17.6,85.4){5} % H-propagator
   \Text(15.7,87.7)[rb]{$H(k_{4})$}
   \Vertex(77.7,64.7){3} % g-g-g vertex
   \Vertex(77.7,35.3){3} % g-g-g vertex
   \Vertex(42.3,64.7){3} % g-g-H vertex
   \Vertex(42.3,35.3){3} % g-g-H vertex
   \Gluon(77.7,35.3)(77.7,64.7){3}{6} % g-propagator
   \Text(74.7,50.0)[rt]{$g$}
   \Gluon(42.3,64.7)(77.7,64.7){3}{7} % g-propagator
   \Text(60.0,67.7)[rb]{$g$}
   \Gluon(42.3,35.3)(77.7,35.3){3}{7} % g-propagator
   \Text(60.0,38.3)[rb]{$g$}
   \Gluon(42.3,35.3)(42.3,64.7){3}{6} % g-propagator
   \Text(39.3,50.0)[rt]{$g$}
\end{picture}
\\
{\sl Diagram~39}\\
$S^\prime=S_{Q\to q-(k2)}$, $\mathrm{rk}=6$
\end{center}
\end{minipage}}

\end{longtable}

%---#] loop diagram39:

\subsection{Group~2 (4-Point)}
\subsubsection*{General Information}
The maximum effective rank in this group is~6.

\begin{subequations}
\begin{align}
r_{1} &= -k_{3}-k_{4}\\
r_{2} &= -k_{4}\\
r_{3} &= 0\\
r_{4} &= -k_{2}
\end{align}
\end{subequations}

\begin{equation}
S=\left(\begin{array}{cccc}
   0&
   S_{1,2}&
   S_{1,3}&
   0\\
   S_{2,1}&
   0&
   S_{2,3}&
   S_{2,4}\\
   S_{3,1}&
   S_{3,2}&
   0&
   0\\
   0&
   S_{4,2}&
   0&
   0\end{array}\right)
\end{equation}
\begin{subequations}
\begin{align}
   S_{1,2}&=m_H^2\\
   S_{1,3}&=s_{12}\\
   S_{2,3}&=m_H^2\\
   S_{2,4}&=-s_{23}-s_{12}+2m_H^2
\end{align}
\end{subequations}

\subsubsection{Diagrams (1)}\begin{longtable}{cc}
\endfirsthead
\endhead
%---#[ loop diagram38:
\index{Diagram0000000038=Diagram 38 (Group 2)}
\hbox{
\begin{minipage}{0.45\textwidth}
\begin{center}
% Diagram 38:
\begin{picture}(140,120)(-10,-10)
   \Gluon(102.4,85.4)(77.7,64.7){3}{6} % g-propagator
   \Text(104.3,87.7)[lb]{$g(k_{1})$}
   \Gluon(102.4,14.6)(77.7,35.3){3}{6} % g-propagator
   \Text(100.5,16.9)[lt]{$g(k_{2})$}
   \DashLine(42.3,64.7)(17.6,85.4){5} % H-propagator
   \Text(15.7,87.7)[rb]{$H(k_{3})$}
   \DashLine(42.3,35.3)(17.6,14.6){5} % H-propagator
   \Text(19.5,16.9)[rt]{$H(k_{4})$}
   \Vertex(77.7,64.7){3} % g-g-g vertex
   \Vertex(77.7,35.3){3} % g-g-g vertex
   \Vertex(42.3,64.7){3} % g-g-H vertex
   \Vertex(42.3,35.3){3} % g-g-H vertex
   \Gluon(77.7,35.3)(77.7,64.7){3}{6} % g-propagator
   \Text(74.7,50.0)[rt]{$g$}
   \Gluon(42.3,64.7)(77.7,64.7){3}{7} % g-propagator
   \Text(60.0,67.7)[rb]{$g$}
   \Gluon(42.3,35.3)(77.7,35.3){3}{7} % g-propagator
   \Text(60.0,38.3)[rb]{$g$}
   \Gluon(42.3,35.3)(42.3,64.7){3}{6} % g-propagator
   \Text(39.3,50.0)[rt]{$g$}
\end{picture}
\\
{\sl Diagram~38}\\
$S^\prime=S_{Q\to q-(k2)}$, $\mathrm{rk}=6$
\end{center}
\end{minipage}}

\end{longtable}

%---#] loop diagram38:


\printindex

\section{Related Work}
If you publish results obtained by using this matrix element code
please cite the appropriate papers in the bibliography of this document.

Scientific publications prepared using the present version of
\textsc{GoSam} or any modified version of it or any code linking to
\textsc{GoSam} or parts of it should make a clear
reference to the publications~\cite{Cullen:2014yla,Cullen:2011ac}.

For graph generation we use QGraf~\cite{Nogueira:1991ex}.
The Feynman diagrams are further processed with the symbolic manipulation
program FORM~\cite{Kuipers:2012rf,Vermaseren:2000nd} using the FORM library
SPINNEY~\cite{Cullen:2010jv}.
The Fortran~90 code is generated using
FORM~\cite{Kuipers:2012rf,Vermaseren:2000nd}.
For the reduction of the tensor integrals
the code uses an implementation of the Laurent series expansion
method~\cite{Mastrolia:2012bu}
from the library Ninja~\cite{Peraro:2014cba}.


Please, make sure, you also give credit to the authors of the scalar
loop libraries, if you configured the amplitude code such that it calls
other libraries than the ones mentioned so far. Depending on your
configuration you might use one or more of the following programs for
the evaluation of the scalar integrals:
\begin{itemize}
\item OneLOop~\cite{vanHameren:2010cp},
\item QCDLoop~\cite{Ellis:2007qk}, which uses FF~\cite{vanOldenborgh:1990yc},
\item LoopTools~\cite{Hahn:1998yk}, which uses FF~\cite{vanOldenborgh:1990yc}.
\item GOLEM95~\cite{Binoth:2008uq,Guillet:2013msa} which uses OneLOop~\cite{vanHameren:2010cp}
   and may be configured such that it uses
   LoopTools~\cite{Hahn:1998yk,vanOldenborgh:1990yc}.
\end{itemize}

\begin{thebibliography}{ABC}
%\cite{Cullen:2014yla}
\bibitem{Cullen:2014yla}
  G.~Cullen, H.~van Deurzen, N.~Greiner, G.~Heinrich, G.~Luisoni, P.~Mastrolia, E.~Mirabella and G.~Ossola {\it et al.},
  ``GoSam-2.0: a tool for automated one-loop calculations within the Standard Model and beyond,''
  Eur.\ Phys.\ J.\ C {\bf 74} (2014) 8,  3001
  [\href{http://arxiv.org/abs/1404.7096}{arXiv:1404.7096 [hep-ph]}].
  %%CITATION = ARXIV:1404.7096;%%
%\cite{Cullen:2011ac}
\bibitem{Cullen:2011ac}
  G.~Cullen, N.~Greiner, G.~Heinrich, G.~Luisoni, P.~Mastrolia, G.~Ossola, T.~Reiter and F.~Tramontano,
  ``Automated One-Loop Calculations with GoSam,''
  Eur.\ Phys.\ J.\ C {\bf 72} (2012) 1889
  [\href{http://arxiv.org/abs/1111.2034}{arXiv:1111.2034 [hep-ph]}].
  %%CITATION = ARXIV:1111.2034;%%
%\cite{Nogueira:1991ex}
\bibitem{Nogueira:1991ex}
  P.~Nogueira,
  ``Automatic Feynman graph generation,''
  J.\ Comput.\ Phys.\  {\bf 105} (1993) 279.
  %%CITATION = JCTPA,105,279;%%
%\cite{Kuipers:2012rf}
\bibitem{Kuipers:2012rf}
  J.~Kuipers, T.~Ueda, J.~A.~M.~Vermaseren and J.~Vollinga,
  ``FORM version 4.0,''
  Comput.\ Phys.\ Commun.\  {\bf 184} (2013) 1453
  [\href{http://arxiv.org/abs/1203.6543}{arXiv:1203.6543 [cs.SC]}].
  %%CITATION = ARXIV:1203.6543;%%
%\cite{Vermaseren:2000nd}
\bibitem{Vermaseren:2000nd}
  J.~A.~M.~Vermaseren,
  ``New features of FORM,''
  arXiv:math-ph/0010025.
  %%CITATION = MATH-PH/0010025;%%
%\cite{Cullen:2010jv}
\bibitem{Cullen:2010jv}
  G.~Cullen, M.~Koch-Janusz and T.~Reiter,
  ``spinney: A Form Library for Helicity Spinors,''
  \href{http://arxiv.org/abs/1008.0803}{arXiv:1008.0803 [hep-ph]}.
  %%CITATION = ARXIV:1008.0803;%%
%\cite{Reiter:2009ts}
%\cite{Peraro:2014cba}
\bibitem{Peraro:2014cba}
  T.~Peraro,
  ``Ninja: Automated Integrand Reduction via Laurent Expansion for One-Loop Amplitudes,''
  Comput.\ Phys.\ Commun.\  {\bf 185} (2014) 2771
  [\href{http://arxiv.org/abs/1403.1229}{arXiv:1403.1229 [hep-ph]}].
  %%CITATION = ARXIV:1403.1229;%%
%\cite{Mastrolia:2012bu}
\bibitem{Mastrolia:2012bu}
  P.~Mastrolia, E.~Mirabella and T.~Peraro,
  ``Integrand reduction of one-loop scattering amplitudes through Laurent series expansion,''
  JHEP {\bf 1206} (2012) 095
   [Erratum-ibid.\  {\bf 1211} (2012) 128]
  [\href{http://arxiv.org/abs/1203.0291}{arXiv:1203.0291 [hep-ph]}].
  %%CITATION = ARXIV:1203.0291;%%
%\cite{Guillet:2013msa}
\bibitem{Guillet:2013msa}
  J.~P.~Guillet, G.~Heinrich and J.~F.~von Soden-Fraunhofen,
  ``Tools for NLO automation: extension of the golem95C integral library,''
  Comput.\ Phys.\ Commun.\  {\bf 185} (2014) 1828
  [\href{http://arxiv.org/abs/1312.3887}{arXiv:1312.3887 [hep-ph]}].
  %%CITATION = ARXIV:1312.3887;%%
%\cite{Binoth:2008uq}
\bibitem{Binoth:2008uq}
  T.~Binoth, J.~P.~Guillet, G.~Heinrich, E.~Pilon and T.~Reiter,
  ``Golem95: a numerical program to calculate one-loop tensor integrals with up
  to six external legs,''
  Comput.\ Phys.\ Commun.\  {\bf 180} (2009) 2317
  [\href{http://arxiv.org/abs/0810.0992}{arXiv:0810.0992 [hep-ph]}].
  %%CITATION = CPHCB,180,2317;%%
%\cite{Cullen:2011kv}
\bibitem{Cullen:2011kv}
  G.~Cullen, J.~P.~.Guillet, G.~Heinrich, T.~Kleinschmidt, E.~Pilon, T.~Reiter, M.~Rodgers,
  ``Golem95C: A library for one-loop integrals with complex masses,''
  Comput.\ Phys.\ Commun.\  {\bf 182 } (2011)  2276-2284.
  [\href{http://arxiv.org/abs/1101.5595}{arXiv:1101.5595 [hep-ph]}].
%\cite{vanHameren:2010cp}
\bibitem{vanHameren:2010cp}
  A.~van Hameren,
  ``OneLOop: For the evaluation of one-loop scalar functions,''
  [\href{http://arxiv.org/abs/1007.4716}{arXiv:1007.4716 [hep-ph]}].
%\cite{Ellis:2007qk}
\bibitem{Ellis:2007qk}
  R.~K.~Ellis, G.~Zanderighi,
  ``Scalar one-loop integrals for QCD,''
  JHEP {\bf 0802 } (2008)  002.
  [\href{http://arxiv.org/abs/0712.1851}{arXiv:0712.1851 [hep-ph]}].
%\cite{vanOldenborgh:1990yc}
\bibitem{vanOldenborgh:1990yc}
  G.~J.~van Oldenborgh,
  ``FF: A Package to evaluate one loop Feynman diagrams,''
  Comput.\ Phys.\ Commun.\  {\bf 66 } (1991)  1-15.
%\cite{Hahn:1998yk}
\bibitem{Hahn:1998yk}
  T.~Hahn, M.~Perez-Victoria,
  ``Automatized one loop calculations in four-dimensions and D-dimensions,''
  Comput.\ Phys.\ Commun.\  {\bf 118 } (1999)  153-165.
  [hep-ph/9807565].
%\cite{Heinrich:2010ax}
\bibitem{Heinrich:2010ax}
  G.~Heinrich, G.~Ossola, T.~Reiter, F.~Tramontano,
  ``Tensorial Reconstruction at the Integrand Level,''
  JHEP {\bf 1010 } (2010)  105.
  [\href{http://arxiv.org/abs/1008.2441}{arXiv:1008.2441 [hep-ph]}].
\end{thebibliography}
\end{document}
